\section{Relevés expérimentaux}
\subsection{Présentation du dispositf expérimental}
\begin{frame}
\frametitle{Présentation du dispositif expérimental}
Afin de mesurer les chocs subits par le coureur on réalise une semelle particulière dotée de:
\begin{itemize}
\item 4 capteurs de pression
\item Un lecteur de carte SD
\item Un microcontrolleur
\end{itemize}
\end{frame}

\begin{frame}
\frametitle{Le capteur de pression DF9-40}
Le capteur de pression est une résistance de variable de loi non linéaire.
\begin{center}
\includegraphics[scale=0.3]{./figures/cal_00.png}
\end{center}
\end{frame}

\begin{frame}
\frametitle{Carte de contrôle}
Voici la carte avec le microcontroleur, les convertisseurs analogiques-numériques et le lecteur de carte SD:
\includegraphics[width=\textwidth]{./figures/carte_00.jpg}

\end{frame}
\subsection{Calibration du dispositif}
\begin{frame}
\frametitle{Calibration - ajustement de courbe 1/2}
On envisage la fonction suivante: $f_{a,b,c}(x) = \frac{a}{x^b}+c$\\
Notons $x_i$ les valeurs en tension correspondant à un poids mesuré $y_i$.\\
Il faut trouver a,b,c minimisant $ \sum_{i=0}^{n} (f_{a,b,c}(x_i)-y_i)^2$
\end{frame}

\begin{frame}
\frametitle{Calibration - ajustement de courbe 2/2}
On utilise la méthode de la descente de gradient pour trouver une valeur approchée de a,b,c minimisant $C(a,b,c)=\sum_{i=0}^{n} (f_{a,b,c}(x_i)-y_i)^2$.
L'ideal étant d'avoir $\frac{\partial C}{\partial a} (a,b,c) = 0$ $\frac{\partial C}{\partial b} (a,b,c) = 0$ $\frac{\partial C}{\partial c} (a,b,c) = 0$, soit autrement dit $\nabla C = 0$. Pour cela on fixe $\alpha \in \mathbb{R}^+$, le taux d'apprentissage et un seuil $\varepsilon \in \mathbb{R}^+$. On construit la suite $P_k=\begin{pmatrix} a_k \\ b_k \\ c_k \end{pmatrix}$ de la facon suivante:
\begin{itemize}
\item $P_0$ fixé de manière arbitraire
\item $P_{k+1}= P_{k} - \alpha \times \nabla C$
\end{itemize}
Condition d'arrêt: $\nabla C < \varepsilon$
\end{frame}

\begin{frame}
\frametitle{Calibration - résultats 1/2}
Résultat de la méthode des moindres carrés avec la descente de gradient effectuée sur la carte:
\begin{center}
\includegraphics[scale=0.3]{./figures/cal_01.png}
\end{center}
\end{frame}

\begin{frame}
\frametitle{Calibration - résultats 2/2}
Comparaison avec la fonction \texttt{curve\_fit} de \texttt{scipy}:
\begin{center}
\includegraphics[scale=0.3]{./figures/cal_02.png}
\end{center}
\end{frame}

\subsection{Premiers résultats}
\begin{frame}
\frametitle{La semelle}
Voici les capteurs disposés sur la semelle:
\includegraphics[width=\textwidth]{./figures/sem_00.jpg}

\end{frame}

\begin{frame}
\frametitle{Les premiers résultats}
Après avoir couru sur differentes surfaces et avec des chaussures de ville et de course on obtient:
\begin{figure}
\includegraphics[scale=0.17]{./figures/res_01.png}
\centering
\end{figure}
\end{frame}
